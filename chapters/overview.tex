
\chapter{Overview}
In a desktop envirnoment there are at least two actors: the display server, that offer the possibility to draw windows on the screen, and the clients, the applications that require such windows. 
The way these two actors communicate each other is called "window communication protocol". 
In the Linux world, two are available: the legacy X11 protocol and the new Wayland protocol. 
The X11 protocol is very old and has serious design problems that cause performance and security issues. 
It has been around for 30 years, the issues it has today simply was not a concern when it was created. 
Since most of the modern desktop environments are already transitioning from X11 to Wayland, this document will focus more on how the new protocol works. 
What this project want to achieve is the development of a Wayland compositor (display server) built on more modern tecnologies the current Wayland compositor are built on. The major aspect of a Wayland compositor are:
\begin{itemize}
	\item The window logic, so the logic used to manage clients's surfaces positions and status
	\item The system interaction, so the ability to send and receive system events like mouses, keyboards, monitors and so on
	\item The rendering system, so the ability to perform manipulations of the clients's surfaces
	\item The presentation system, so the ability to show the clients's surfaces on the screen
\end{itemize}
\newpage
\begin{center}
	\includegraphics[width=0.9\textwidth]{standard_wayland_compositor.png}
	\captionof{figure}{Standard modern compositor}
\end{center}
On a modern Wayland compositor, the presentation system is handled by the Drm/Kms system. This system has its own api and allow to draw surfaces directly on the screen. It is not a rendering system, so it is not able to perform complex image manipulations, but just present them on the screen in a specific order. For this reason, compositors generally associate a rendering engine (like an OpenGL or OpenGL/ES engine) with the Drm/Kms system, so that with the first performs complex renderings and with the second present the result on the screen. This requires some interaction logic to make the rendering engine and the drm/kms system to synchronize each other and to work properly.\newline
The system interaction is generally handled with two separate events logics depending on the mode the compositor is launched.
\begin{itemize}
	\item Window mode: the compositor is launched as a window inside another display server, like a Xorg server or another Wayland compositor.
	\item Direct mode: the compositor is launched directly on the TTY and work as main display server.
\end{itemize}
Two events logics are required because the entities that generate the events are foundamentally different, so they are very difficult to unify without losing some features of one of the two.
\newpage
\begin{center}
	\includegraphics[width=0.8\textwidth]{wcomp_compositor.png}
	\captionof{figure}{WComp compositor}
\end{center}

This project use different approaches for the presenting system, the rendering system and the system interaction.
The rendering system use the Vulkan api as backend with the ability to draw directly on the screen. This means that the rendering system can be used to draw and present on its own, without interacting with other systems. This translate to a much more simplier logic, while reducing synchronization points and increase the performances.\newline
For the system interaction, a library called "Platform Abstraction Layer" have been developed. This library unify the events generated by the system itself with the events generated by another display server. In this way the compositor can rely on just one events logic and the PAL will translate them appropiately depending on the mode the compositor has been launched.
The whole project is built using Rust, a modern language with unique feature, like the "ownership" memory system that makes the compiler able to detect and avoid (through compile time checks) undefined behaviours, like double free, null pointer exception, data access with wrong alignment and many more. In a mutithreading, this feature avoid to spent days (or weeks) into searching for some unhandled race condition bug or similar.